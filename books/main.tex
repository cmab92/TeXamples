\documentclass[%
twoside,symmetric,	% main question: print-version or online-only?
nols, % suppress letterspacing for small caps text
a4paper, % ...
notoc,	% i want that custom
justified,	% unusual for tufte-book! Anyway ...
nobib, % custom references
]{tufte-book} % 

%% Change page tiling:
\geometry{
	left=24.8mm, % left margin
	textwidth=110mm, % orig: 100mm, main text block
	marginparsep=8.2mm, % orig: 8.2mm, margin between main text block and margin notes
	marginparwidth=44.4mm % orig: 49.4mm, width of margin notes
}

\usepackage{fontspec} % e.g. using different fonts
\usepackage{pgfplots} % :)
\usepackage{amsmath,amssymb,amsfonts,amsthm,mathrsfs,mathtools,xfrac} % Regarding equations
\usepackage[utf8]{luainputenc}	% LuaLaTex!
\usepackage[capitalize,nameinlink,noabbrev]{cleveref} % simplifies \ref usage and improves formatting
\usepackage[framemethod=tikz]{mdframed} % boxes (e.g. for theorems, definitions, etc.)
\usepackage{emoji} % just for fun
\usepackage{lipsum} % blind text
\usepackage{psvectorian} % just for fun
\usepackage{tabto}
\usepackage{dirtytalk}	% mainly \say command
\usepackage{ifthen} % ... e.g. for more complicated plots
\usepackage{romannum}	% ...
\usepackage{array,readarray}	% read arrays
\usepackage{siunitx}	% ...
\usepackage{epigraph}	% just for fun
\usepackage{fouriernc}	% libertine eulervm
\usepackage{verbatim} % for \verbatiminput


% Optional: make index entries smaller

%% Language
\usepackage[ngerman,english]{babel} % mainly for correct word-breaking (the last-mentioned is dominant, e.g. Contents instead of Inhaltsverzeichnis)

%% Regarding bibliography
\usepackage{natbib}
\usepackage{bibentry}
\nobibliography*

%% index (:
%% Links (include after natbib for use in bibliography)
\usepackage{hyperref} % references, citations, URLs, and table of contents clickable in PDF
\usepackage{url}
\usepackage{imakeidx}       % index
% Create index using makeindex
\makeindex[intoc, columns=2, title=Index] % an index with hyperlink is a little more complicated. Yet, Imo, this isnt need, as an index is not needed in a PDF (use ctrl+f)...

%% Regarding pgfplots and tikz
%\usetikzlibrary{external}
%\tikzexternalize[prefix=tikz/]
\usetikzlibrary{calc,arrows,arrows.meta,angles,quotes,plotmarks,fadings,shapes,spy,mindmap,backgrounds}
\usepgfplotslibrary{colorbrewer,fillbetween}
\pgfplotsset{compat=newest}

%% Regarding TOC
\setcounter{secnumdepth}{2} % activate section and chapter numbering
\setcounter{tocdepth}{1} % section and chapter number-depth in TOC

% colormaps
\usepgfplotslibrary{colorbrewer} % cf. https://tikz.dev/pgfplots/libs-colorbrewer
\def\mycolorscheme{RdYlBu}

\pgfplotsset{colormap/\mycolorscheme-11} 

\def\colorforcurvesi{\mycolorscheme-M}
\def\colorforcurvesii{\mycolorscheme-K}
\def\colorforcurvesiii{\mycolorscheme-I}
\def\colorforcurvesiv{\mycolorscheme-E}
\def\colorforcurvesv{\mycolorscheme-C}
\def\cfcA{\mycolorscheme-A}
\def\cfci{\mycolorscheme-B}
\def\cfcii{\mycolorscheme-C}
\def\cfcE{\mycolorscheme-E}
\def\cfciii{\mycolorscheme-F}
\def\cfciv{\mycolorscheme-G}
\def\cfcv{\mycolorscheme-H}
\def\cfcvi{\mycolorscheme-I}
\def\cfcvii{\mycolorscheme-J}
\def\cfcviii{\mycolorscheme-L}
\def\cfcix{\mycolorscheme-N}

\def\posmatcolor{\mycolorscheme-B}
\def\negmatcolor{\mycolorscheme-N}

%
\def\mylocalcolorproof{\mycolorscheme-I}
\def\mylocalcolordef{\mycolorscheme-J}
\def\mylocalcolortheo{\mycolorscheme-N}

\def\urlcolor{\mycolorscheme-O}
\def\linkcolor{\mycolorscheme-O}
\def\citecolor{\mycolorscheme-A}

\gdef\mysidenotetextcolor{\mycolorscheme-N}
\gdef\mycitationtextcolor{\mycolorscheme-B}
\gdef\mychaptertitlecolor{\mycolorscheme-O}

\hypersetup{
	linkcolor=\linkcolor,
	citecolor=\citecolor,
	urlcolor=\urlcolor,
	colorlinks=true,
}


\gdef\mymathcolor{black}
\everymath{\color{\mymathcolor}}       % inline math
\everydisplay{\color{\mymathcolor}}    % display math

\renewcommand\thefootnote{\sffamily\textcolor{\mysidenotetextcolor}{\arabic{footnote}}}

\setsidenotefont{
	\sffamily
	\footnotesize
	\color{\mysidenotetextcolor}
}   

\setcaptionfont{
	\sffamily
	\footnotesize
	\color{\mysidenotetextcolor}
}   

\setmarginnotefont{
	\sffamily
	\small
	\color{\mysidenotetextcolor}
}   

\setcitationfont{
	\sffamily
	\color{\mycitationtextcolor}
}

\gdef\mysidenotefont{\sffamily\textcolor{\mysidenotetextcolor}}
\def\sectionfont{\sffamily\LARGE}

% chapter format
\titleformat{\chapter}%
{\huge\sffamily\scshape\color{\mychaptertitlecolor}}% format applied to label+text
{{\thechapter}\llap}% label
{1em}% horizontal separation between label and title body
{}% before the title body
[]% after the title body

% section format
\gdef\mysectionfont{\sffamily\Large\color{\mychaptertitlecolor}}
\titleformat{\section}%
{\mysectionfont}% !!!!!!!!!! % \itshape % format applied to label+text
{{\thesection}\llap}% {\llap{\colorbox{white}{\parbox{1.5cm}{\hfill\color{white}\thesection}}}}% label
{1em}% horizontal separation between label and title body
{}% before the title body
[]% after the title body

% subsection format
\titleformat{\subsection}%
{\sffamily\large\itshape\color{\mychaptertitlecolor}}%  % 
{{\thesubsection}\llap}% {\llap{\colorbox{white}{\parbox{1.5cm}{\hfill\color{white}\thesubsection}}}}% label
{1em}% horizontal separation between label and title body
{}% before the title body
[]% after the title body


%% alternative fonts
\newfontface\kitschy{ZapfinoExtraLT-Four} % really fancy
\newfontface\kitschytwo{ZapfinoForteLTPro} % really fancy
\newfontfamily\Examplefonti{Tosca-8Rln} % or any elegant font
\newfontfamily\Examplefontii{Medievalalphabet-4EY6} % or any elegant font
\newfontfamily\Examplefontiii{SpanishArmyShieldsTwo-g9aE} % or any elegant font
\newfontfamily\Examplefontiv{Hansschoenspergerrandomish-VW6B} % or any elegant font
\newfontfamily\reasonablefont{EB Garamond} % or 

\makeatletter
\newcommand\chapterauthor[1]{#1\gdef\@chapterauthor{#1}}
\def\@chapterauthor{}
\fancypagestyle{mystyle}{%
	\fancyhf{}%
	\renewcommand{\chaptermark}[1]{\markboth{##1}{}}%
	\fancyhead[LE]{\thepage\quad{\newlinetospace{\leftmark}}}% 
	\fancyhead[RO]{\smallcaps{\newlinetospace{\@chapterauthor}}\quad\thepage}%
}
\fancypagestyle{mysecondstyle}{% not needed here ...
	\fancyhf{}%
	\renewcommand{\chaptermark}[1]{\markboth{##1}{}}%
	\fancyhead[LE]{\thepage\quad{\newlinetospace{\leftmark}}}% 
	\fancyhead[RO]{\MakeUppercase{\newlinetospace{\@chapterauthor}}\quad\thepage}%
}
\makeatother

\input{settings/usebibentry.tex}

%\newcommand{\mycite}[1]{[\citeauthorname{#1}, \citepublicationyear{#1}]}
%\newcommand{\mycitepage}[2]{[\citeauthorname{#1}, \citepublicationyear{#1}], p. {#2}}

\newcommand{\mycite}[1]{\cite{#1}}
\newcommand{\mycitepage}[2]{\cite{#1}, p. {#2}}

\newcommand{\citeauthorname}[1]{\citeauthor{#1}}
\newcommand{\citepublicationyear}[1]{{\usebibentry{#1}{year}}}
\newcommand{\citesay}[1]{\hfill \\---\citeauthor{#1}, \usebibentry{#1}{title}, \usebibentry{#1}{year}}

%\newcommand{\citeintextauthorname}[1]{\citeauthor{#1}\fullsidenotecite{#1}}
%\newcommand{\fullsidenotecite}[1]{{\citesquarebrackets{#1}}}
%\newcommand{\citepage}[2]{\citesquarebracketspage{#1}{#2}}
%\newcommand{\sidecitepage}[2]{\citesquarebracketspage{1}{2}}
%\newcommand{\citepublicationtitle}[1]{``{\usebibentry{#1}{title}}''}




%\newcommand{\fullsidenotecite}[1]{{\cite{#1}}}
%\newcommand{\citepage}[2]{\citeauthor{#1}, p. {#2}}
%\newcommand{\sidecitepage}[2]{\footnote{\citealt{#1}, p. {#2}}}
%\newcommand{\citesay}[1]{\hfill \\---\citeauthor{#1}, \usebibentry{#1}{title}}
%\newcommand{\citeauthorname}[1]{\citeauthor{#1}}
%\newcommand{\citepublicationtitle}[1]{``{\usebibentry{#1}{title}}''}
%\newcommand{\citepublicationyear}[1]{{\usebibentry{#1}{year}}}
%\newcommand{\citeintextauthorname}[1]{\citeauthor{#1}\fullsidenotecite{#1}}
%\newcommand{\citesquarebrackets}[1]{[\citeauthorname{#1} (\citepublicationyear{#1})]}
%\newcommand{\citesquarebracketspage}[2]{[\citeauthorname{#1} (\citepublicationyear{#1})], p. {#2}}




















\def\myltab{\tabto{4cm}} % glossary and nomenclature
\def\goldenratio{1.6180339887} % just for fun

\newcommand{\ie}{\textit{i.e.,}~}
\newcommand{\cf}{\textit{cf.}~}
\newcommand{\Cf}{\textit{Cf.}~}
\newcommand{\eg}{\textit{e.g.,}~}
\newcommand{\Eg}{\textit{E.g.,}~}
\newcommand{\sic}{[\textit{sic}]}
\newcommand{\wrt}{w.r.t.~}
% theorem

%\def\mylocalcolor{white}
\def\mylocalopacval{20}
\def\mylocalopacvalproof{10}
\def\mylocalopacvaldef{20}
\def\mylocalopacvaltheo{30}
\def\mylocaltopskip{1cm}
%%%%%%%%%%%%%%%%%%%%%%%%%%%%%%%%%%%%%%%%%%%%%%%%%%%%%%%%%%%%%%%%%%%%%%%%%%%%%%%%%%%%%%%%%%%%%%%%%%%%%
\makeatletter
\DeclareDocumentCommand{\mdtheorem}{ O{} m o m o }%
{\ifcsdef{#2}%
	{\mdf@PackageWarning{Environment #2 already exits\MessageBreak}}%
	{%
		\IfNoValueTF {#3}%
		{%#3 not given -- number relationship
			\IfNoValueTF {#5}%
			{%#3+#5 not given
				\@definecounter{#2}%
				\expandafter\xdef\csname the#2\endcsname{\@thmcounter{#2}}%
				\newenvironment{#2}[1][]{%
					\refstepcounter{#2}%
					\ifstrempty{##1}%
					{\let\@temptitle\relax}%
					{%
						\def\@temptitle{\mdf@theoremseparator%
							\mdf@theoremspace%
							\mdf@theoremtitlefont%
							##1}%
						\mdf@thm@caption{#2}{{#4}{\csname the#2\endcsname}{##1}}%
					}%
					\begin{mdframed}[#1,frametitle={\strut#4\ \csname the#2\endcsname%
							\@temptitle}]}%
					{\end{mdframed}}%
				\newenvironment{#2*}[1][]{%
					\ifstrempty{##1}{\let\@temptitle\relax}{\def\@temptitle{\mdf@theoremseparator\ ##1}}% <- the problem was here
					\begin{mdframed}[#1,frametitle={\strut#4\@temptitle}]}%
					{\end{mdframed}}%
			}%
			{%#5 given -- reset counter
				\@definecounter{#2}\@newctr{#2}[#5]%
				\expandafter\xdef\csname the#2\endcsname{\@thmcounter{#2}}%
				\expandafter\xdef\csname the#2\endcsname{%
					\expandafter\noexpand\csname the#5\endcsname \@thmcountersep%
					\@thmcounter{#2}}%
				\newenvironment{#2}[1][]{%
					\refstepcounter{#2}%
					\ifstrempty{##1}%
					{\let\@temptitle\relax}%
					{%
						\def\@temptitle{\mdf@theoremseparator%
							\mdf@theoremspace%
							\mdf@theoremtitlefont%
							##1}%
						\mdf@thm@caption{#2}{{#4}{\csname the#2\endcsname}{##1}}%
					}
					\begin{mdframed}[#1,frametitle={\strut#4\ \csname the#2\endcsname%
							\@temptitle}]}%
					{\end{mdframed}}%
				\newenvironment{#2*}[1][]{%
					\ifstrempty{##1}%
					{\let\@temptitle\relax}%
					{%
						\def\@temptitle{\mdf@theoremseparator%
							\mdf@theoremspace%
							\mdf@theoremtitlefont%
							##1}%
						\mdf@thm@caption{#2}{{#4}{\csname the#2\endcsname}{##1}}%
					}%
					\begin{mdframed}[#1,frametitle={\strut#4\@temptitle}]}%
					{\end{mdframed}}%
			}%
		}%
		{%#3 given -- number relationship
			\global\@namedef{the#2}{\@nameuse{the#3}}%
			\newenvironment{#2}[1][]{%
				\refstepcounter{#3}%
				\ifstrempty{##1}%
				{\let\@temptitle\relax}%
				{%
					\def\@temptitle{\mdf@theoremseparator%
						\mdf@theoremspace%
						\mdf@theoremtitlefont%
						##1}%
					\mdf@thm@caption{#2}{{#4}{\csname the#2\endcsname}{##1}}%
				}
				\begin{mdframed}[#1,frametitle={\strut#4\ \csname the#2\endcsname%
						\@temptitle}]}%
				{\end{mdframed}}%
			\newenvironment{#2*}[1][]{%
				\ifstrempty{##1}{\let\@temptitle\relax}{\def\@temptitle{:\ ##1}}%
				\begin{mdframed}[#1,frametitle={\strut#4\@temptitle}]}%
				{\end{mdframed}}%
		}%
	}%
}
\makeatother
%%%%%%%%%%%%%%%%%%%%%%%%%%%%%%%%%%%%%%%%%%%%%%%%%%%%%%%%%%%%%%%%%%%%%%%%%%%%%%%%%%%%%%%%%%%%%%%%%%%%%%%%%%%%%%%%%%%%%%%%%%%%%%%%%%%%%%%%%%%%%%%%%
\newenvironment{myoutline}
{\mdfsetup{
		frametitle={},
		%		innertopmargin=10pt,
		%		frametitleaboveskip=-\ht\strutbox,
		%		frametitlealignment=\center
	}
	\begin{mdframed}%
	}
	{\end{mdframed}}
%%%%%%%%%%%%%%%%%%%%%%%%%%%
\mdfdefinestyle{myStyle}{%
	backgroundcolor=\mylocalcolor!\mylocalopacval, 
	nobreak=true,
	linewidth=0pt,
	roundcorner=5pt,
	innertopmargin=-0.5ex,
}
\mdfdefinestyle{myproofStyle}{%
	%	backgroundcolor=\mylocalcolorproof!\mylocalopacvalproof, 
	nobreak=false,
	linewidth=0pt,
	roundcorner=5pt,
	innertopmargin=-0.5ex,
	theoremseparator={ ---},
	theoremspace=\space
}
\mdfdefinestyle{mydefStyle}{%
	backgroundcolor=\mylocalcolordef!\mylocalopacvaldef, 
	nobreak=true,
	linewidth=0pt,
	roundcorner=5pt,
	innertopmargin=-0.5ex,
}
\mdfdefinestyle{mytheoStyle}{%
	backgroundcolor=\mylocalcolortheo!\mylocalopacvaltheo, 
	nobreak=true,
	linewidth=0pt,
	roundcorner=5pt,
	innertopmargin=-0.5ex,
}

\mdtheorem[style=mydefStyle]{mydef}{Definition}
\numberwithin{mydef}{chapter}

\mdtheorem[style=myproofStyle]{myproof}{Proof}
\numberwithin{myproof}{chapter}

\mdtheorem[style=mytheoStyle]{mytheo}{Theorem}
\numberwithin{mytheo}{chapter}

\makeatletter
\if@cref@capitalise
\crefname{mydef}{Definition}{Definition}
\crefname{mytheo}{Theorem}{Theorem}
\crefname{myproof}{Proof}{Proof}
\else
\crefname{mydef}{definition}{definition}
\crefname{mytheo}{theorem}{theorem}
\crefname{myproof}{proof}{proof}
\fi
\makeatother

%

% proofs
\newtheorem{myshortproof*}{Proof}
\numberwithin{myshortproof*}{chapter} % important bit
% examples
\newtheorem{myexample}{Example}
\numberwithin{myexample}{chapter} % important bit
% lemma
\newtheorem{mylemma}{Lemma}
\numberwithin{mylemma}{chapter} % important bit



%\usepackage{xpatch}
%\xapptocmd{\appendix}{%
	%	\numberwithin{mytheo}{section}
	%	\numberwithin{mydef}{section}
	%	\numberwithin{myproof}{section}
	%	\numberwithin{mylemma}{section}
	%	\numberwithin{myexample}{section}
	%}{\typeout{Success}}{}

% title
\newcommand{\titleofthesis}{Der lustige Titel meiner Publikation}
\newcommand{\meinetolleuni}{Universität Hinterdupfing}
\newsavebox{\hslogo}
\savebox{\hslogo}[230mm]{
	\begin{tikzpicture}[overlay]
		\def\widthornament{4}
		\node (orig) at (1,7) {};
		\node[anchor=center] (2) at (orig) {
			\begin{pspicture}(0,0)(\widthornament,\widthornament)% 
				% pstricks in pgfplots
				% (x0,x1)(y0,y1), where x and y are lower left and upper right corner 
				\rput[tl](0,0){\psvectorian[width=\widthornament cm]{91}} % https://ftp.rrze.uni-erlangen.de/ctan/graphics/pstricks/contrib/pst-vectorian/doc/psvectorian.pdf
			\end{pspicture}
		};
		\node at ($(2.south) + (0.6,-2.7)$) {\small\kitschy \meinetolleuni};
	\end{tikzpicture}
}

\title[\titleofthesis]{%
	\usebox{\hslogo}
	\setlength{\parindent}{0pt}%
	\vspace{2cm} \par \Huge Der lustige Titel\\ meiner Publikation \par \vspace{0.5cm} 
	\small{
		Dissertation zur Erlangung des Doktorgrades	\textit{Dr. rer. tex.}\\
		der Fakultät für Irgendwas und Noch Etwas\\
		der \meinetolleuni\\
		\vspace{1cm} \\
		\textit{von} Irgend R. Jemanden aus Hinterdupfing $\quad$ | $\quad$ 2025 \par
		\textit{\meinetolleuni,} \textit{Institut für Form ohne Inhalt}
	}
}


\begin{document}
	% Frontmatter
	
\maketitle% this prints the handout title, author, and date

% Second page
\setlength{\parindent}{0pt}%
\thispagestyle{empty}
This document was typeset in \LaTeX~using the tufte-book document class. 
\vspace{2cm}

This document is licensed under the 
\href{https://creativecommons.org/licenses/by/4.0/}{Creative Commons Attribution 4.0 International License (CC BY 4.0)}.

This work was created with \LaTeX using the tufte-book document class., hyperref, cleveref, glossaries, etc.
All packages are used under their respective licenses.

Forwarding, publishing, or using this material for any purpose other than the intended internal use is not permitted.  
All rights reserved.
\vfill
Christopher Bonenberger (2025) \titleofthesis \clearpage
% Third page
\setlength{\parindent}{0pt}%
\thispagestyle{empty}
\vfill

%Tag der Promotion: 
% Fourth page
\setlength{\parindent}{0pt}%
\newpage\null

% Abstract
\thispagestyle{empty}
\section*{Abstract}
%\lipsum[1]
\input{../src/texts/abstract.tex}
% 6. page
\setlength{\parindent}{0pt}%
\newpage\null\thispagestyle{empty}\newpage

% thanks to
%\setlength{\parindent}{0pt}%
%\thispagestyle{empty}
%\hfill %  Dedicated to ...
%\pagenumbering{roman}
% 8. page
%\setlength{\parindent}{0pt}%
%\newpage\null\thispagestyle{empty}\newpage

	% Nomenclature & Glossary
	\section*{Nomenclature}\label{nomenclature}
\thispagestyle{empty}


% As a start:
\def\mylhypertarget{rkhsnomen}
\newcommand{\hilbertspace}{\hypersetup{linkcolor=\mymathcolor}\hyperlink{\mylhypertarget}{\mathcal{H}}}%
{$\hilbertspace$ \myltab \hypertarget{\mylhypertarget}{Hilbert space of functions} \par}

\def\mylhypertarget{eulernomen}
\newcommand{\euler}{\hypersetup{linkcolor=\mymathcolor}\hyperlink{eulernomen}{\text{e}}}%
{$\euler$ \myltab \hypertarget{\mylhypertarget}{Euler's number (mathematical constant)} \par}

\def\mylhypertarget{complexnrnomen}
\newcommand{\complexunit}{\hypersetup{linkcolor=\mymathcolor}\hyperlink{\mylhypertarget}{\mathfrak{i}}}%
{$\complexunit$ \myltab \hypertarget{\mylhypertarget}{Complex unit ($\complexunit^2=-1$)} \par}

% an easier version would be:
%\newcommand{\hilbertspace}{\mathcal{H}%
%{$\hilbertspace$ \tab Hilbert space of functions \par}

\newcommand{\myendofproof}{\hfill\text{${\Box} $}}
\newcommand{\myendofproofequation}{\hfill\text{$\tag*{\Box} $}}

\cleardoublepage
	\cleardoublepage
\section*{Glossary}
\thispagestyle{empty}
\newacronym{baan}{BAAN}{Benevolent Artificial Anti-Natalism}
\newglossaryentry{sentience}{
	name=sentience,
	description={capacity to have subjective experiences or feelings}
}
	
	
	%% TOC
	\sffamily % sans serif
	\tableofcontents
	\normalfont % back to normal
	\cleardoublepage
	
	% Document body
	\bibinput{../src/para/references}
	\pagestyle{mystyle}	
	\setcounter{page}{1}
	\pagenumbering{arabic}
	
	\chapter{Introduction}\label{sec:introduction}
		\epigraph{I would prefer not to [...]}{\citesay{melville1980bartleby}}
\lipsum
		
	\chapter{TikZ and PGFplots\,--\,Basics}
		\section{The axis-environment}
			An natural point to start using pgfplots is the axis-environment\index{axis}. We begin with the following simple example, which generates \cref{fig:simplepgfexample1}.

\verbatiminput{../src/figures/simplepgfexample1.tex}

\begin{figure}[h]
	\centering
	\begin{tikzpicture}
	\begin{axis}[
		xlabel=$x$, ylabel={$f(x)$},
		]
		\addplot {exp(-x^2)*sin(deg(x))};
	\end{axis}
\end{tikzpicture}
	\caption{This is fine for a start.}
	\label{fig:simplepgfexample1}
\end{figure}
Of course, we still want to improve this in many ways. We begin with \emph{global} versus \emph{local} styles.

\newthought{Global styles} are default in the sense that they apply to all axes and plots in the document (unless overridden). Thus, the following 
\begin{verbatim}
	\pgfplotsset{
		every axis/.style = {
			axis lines=center,
			width=\textwidth, height=\textwidth/\goldenratio,
			grid=both,
		},
		every axis plot/.style = {
    		ultra thick, mark=*
		}
	}
\end{verbatim}

\pgfplotsset{
	every axis/.style = {
		axis lines=center,
		width=\textwidth, height=\textwidth/\goldenratio,
		grid=both,
	},
	every axis plot/.style = {
		ultra thick, mark=*
	}
}

makes \cref{fig:simplepgfexample1} looking like \cref{fig:simplepgfexample1withdefault}.

\begin{figure}[h]
	\centering
	\begin{tikzpicture}
	\begin{axis}[
		xlabel=$x$, ylabel={$f(x)$},
		]
		\addplot {exp(-x^2)*sin(deg(x))};
	\end{axis}
\end{tikzpicture}
	\caption{Same as \cref{fig:simplepgfexample1} but with global styles.}
	\label{fig:simplepgfexample1withdefault}
\end{figure}

\newthought{Cycle lists} can be pretty handy:

\begin{verbatim}
	\pgfplotsset{
		cycle list={
			{\colorforcurvesi, thick},
			{\colorforcurvesii, thick},
			{\colorforcurvesiii, thick},
			{\colorforcurvesiv, thick},
			{\colorforcurvesv, thick},
			{\colorforcurvesi, dashed},
			{\colorforcurvesii, dashed},
			{\colorforcurvesiii, dashed},
			{\colorforcurvesiv, dashed},
			{\colorforcurvesv, dashed},
		}
	}
\end{verbatim}

\begin{figure}[h]
	\centering
	\begin{tikzpicture}
	\begin{axis}[
		xlabel=$x$, ylabel={$f(x)$},
		]
		\addplot {exp(-x^2)*sin(deg(x))};
	\end{axis}
\end{tikzpicture}
	\caption{Same as \cref{fig:simplepgfexample1cycle} with more graphs to show cycle list.}
	\label{fig:simplepgfexample1withdefault}
\end{figure}

\pgfplotsset{
	cycle list={
		{\colorforcurvesi, thick},
		{\colorforcurvesii, thick},
		{\colorforcurvesiii, thick},
		{\colorforcurvesiv, thick},
		{\colorforcurvesv, thick},
		{\colorforcurvesi, dashed},
		{\colorforcurvesii, dashed},
		{\colorforcurvesiii, dashed},
		{\colorforcurvesiv, dashed},
		{\colorforcurvesv, dashed},
	}
}



Local styles are applied inside a specific axis or \verb*|\addplot| and override global settings.

%\verbatiminput{../src/figures/simplepgfexample2.tex}

\tikzsetnextfilename{simplepgfexample2}
\begin{figure}[h]
	\centering
	\begin{tikzpicture}
	\begin{axis}[
		axis lines=center,
		width=\textwidth, height=\textwidth/\goldenratio,
		xlabel=$x$, ylabel={$f(x)$},
		grid=both,
		ymax=1.1,
		ymin=-1.1,
		]
		\addplot[
			domain=-3.2:3.2,
			samples=101, 
			line width=3,
			\colorforcurvesi]
			{exp(-x^2)*sin(deg(4*x))};
	\end{axis}
\end{tikzpicture}
	\caption{This one is a little more sophisticated.}
	\label{fig:simplepgfexample2}
\end{figure}

Now lets go for several functions:

%\verbatiminput{../src/figures/simplepgfexample3.tex}

\tikzsetnextfilename{simplepgfexample3}
\begin{figure*}[h]
	\centering
	\begin{tikzpicture}
	\begin{axis}[
		axis lines=center,
		width=\textwidth, height=\textwidth/\goldenratio,
		xlabel=$x$, ylabel={$y$},
		grid=both,
		ymax=1.4,
		ymin=-1,
		ytickmin=-0.5,
		ytickmax=1,
		legend style={draw=none,anchor=north east, at={(1,1)}},
		]
		\addplot[
		domain=-3.2:3.2,
		samples=101, 
		line width=3,
		\colorforcurvesv]
		{exp(-(x+1)^2)*sin(deg(3*(x+1)))};
		\addlegendentry{$g(x) = \euler^{-(x+1)^2}\sin\left(3(x+1)\right)$}
		\addplot[
		domain=-3.2:3.2,
		samples=101, 
		line width=3,
		\colorforcurvesi]
		{exp(-(x-1)^2)*sin(deg(4*(x-1)))};
		\addlegendentry{$f(x) = \euler^{-(x-1)^2}\sin\left(4(x-1)\right)$}
	\end{axis}
\end{tikzpicture}
	\caption{This one is a little more sophisticated.}
	\label{fig:simplepgfexample3}
\end{figure*}


	\chapter[Quotes \& References]{Quotations, Citations and \\ References of All Kinds}
		We begin with some technicalities. 
		\section[References]{References of All Kinds}
		The basic structures for citations\index{citation}, references\index{references}, and the like are defined in the following files:
\begin{itemize}
	\item \texttt{settings/colorsettings.tex} \\ 
	Among other things, this file defines the colors for all the different types of references. For example, this reference to \cref{sec:introduction} has the color \linebreak \verb|\linkcolor| (\cf \cref{fig:linkcolor}) due to:
	\begin{marginfigure}[0cm]
		\centering
		\begin{tikzpicture}
			\draw[fill=\linkcolor] (0, 0) rectangle ++(\marginparwidth,\marginparwidth/\goldenratio); 
		\end{tikzpicture}\caption{The color \texttt{\linkcolor}.}\label{fig:linkcolor}
	\end{marginfigure}
	\begin{verbatim}
		\def\mycolorscheme{RdYlBu}
		...
		\def\linkcolor{\mycolorscheme-O}
		...
		\hypersetup{
			linkcolor=\linkcolor,
			citecolor=\citecolor,
			urlcolor=\urlcolor,
			colorlinks=true,
		}
	\end{verbatim}
	\item \texttt{settings/usebibentry.tex} \\ 
	A relatively messy way to provide a programmable way to pull BibTeX info into the text (not necessarily as a formal reference).
	
	\item \texttt{../src/para/citecommands.tex} \\ 
	Some more or less useful commands for citations and different kinds of quotes.
	
	\item \texttt{../src/para/nomenclature.tex} \\ 
	In the simplest case, one would use such a file just to throw in all the \verb|\newcommand| for formula symbols, and the like. In this example, however, we go a bit further and equip everything with the corresponding hyperlinks and also generate a nomenclature. This allows the following\sidenote{Margin notes are also references and are assigned certain properties (font, color, \dots) in the aforementioned files. Just like references to equations or similar items. For example, a reference to the most beautiful equation in mathematics: \cref{eq:mostbeautifulequation}. Is it?}
	\begin{equation}\label{eq:mostbeautifulequation}
		\euler^{\complexunit\pi} - 1 = 0.
	\end{equation}
	Note that the symbols are linked to the nomenclature. Everything in the nomenclature is hand-crafted, although there are certainly various ready-made solutions available.
	
	\item \texttt{../src/para/glossary.tex} \\ 
	Pretty much the same as for the nomenclature. Simply lets try by referencing to an \RKHS. Of course, we may also set the \verb|\hypertarget| to somewhere else (\eg to a corresponding mathematical definition).
	
	\item \texttt{../src/para/references} \\ 
	Here, we add (literature) sources. Since the resulting section does not automatically appear in the \mytoc~ we need
	\begin{verbatim}
		\phantomsection
		\addcontentsline{toc}{chapter}{Bibliography}
	\end{verbatim}
	followed by
	\begin{verbatim}
		\bibliographystyle{plainnat}
		\bibliography{../src/para/references}
	\end{verbatim}
	Notably, \emph{before} using any literature-reference use\\
	\verb|\bibinput{../src/para/references}|\\
	within the \verb|document|.
	
\end{itemize}

So far, so good. Let’s turn to the next topic: citations. 
		\section[The Cost of Ink?]{The Cost of Ink?}
			\epigraph{A designer knows he has achieved perfection not when there is nothing left to add, but when there is nothing left to take away.}{\citesay{de1992wind}}

The following text is mainly an example regarding citation. I haven’t actually read the book \say{\citesourcename{de1992wind}} by \citeauthorname{de1992wind}. Instead the above quote is taken from the \href{https://tug.ctan.org/macros/latex/contrib/tufte-latex/sample-book.pdf}{Tufte-LaTeX sample book}. The quote basically says that good design is reached through simplicity and clarity, not by adding more features or decoration. A design is complete when everything unnecessary has been removed, leaving only what is essential and functional. Although this seems to be a nice guideline regarding the \say{design} of scientific texts and figures, there are other opinions on the topic of design in the broadest sense. One particularly striking example:
\myepigraphintext{quote:hundertwasser}{Die gerade Linie ist ein wahres Werkzeug des Teufels. Wer sich ihrer bedient, hilft mit am Untergang der Menschheit.}{hundertwasserweb2}
In fact, \citeauthorname{hundertwasserweb1} is a stark contrast to \say{modern} conceptions of design. It is not only that he dislikes straight lines a lot\footnote{\Cf \say{\citesourcename{hundertwasserweb1}} (\cite{hundertwasserweb1}).} a lot, but his conception of design is generally somewhat opposing the one that is expressed by the introductory quote. His paintings and especially his texts provoke, yet at times they also display a great sense of humor (see for example \cite{hundertwasserweb3}). At first glance, these statements, which seem almost anarchistic, do not appear to fit the given context. The connection, however, is made by a contemporary from the same time and country.
	\chapter[Anything Goes]{Anything Goes}
		\epigraph{Die wissenschaftliche Ausbildung [...] simplifiziert die \frqq Wissenschaft\flqq, indem sie die Akteure simplifiziert.}{\citesay{feyerabend1986wider}}

We don’t want to be accused of that! Instead, let’s take this statement as an excuse to dive into the \say{fun} of \verb|pgfplots|... \emoji{woozy-face}


A very powerful tool for dealing with more complex figures are \verb|foreach|-loops\index{for loops}. For example, these may be used to plot families of curves as shown in \cref{fig:simplepgfexample4}. Some smaller adjustments (opacity\index{opcaity} and a legend\index{axis!legend}) are shown in \cref{fig:simplepgfexample5,fig:simplepgfexample6}. These figures are based on \verb|edef| (expanded definition).\footnote{There is a \href{https://tex.stackexchange.com/questions/17638/pgfplots-foreach-equivalent-to-tikzs-with-multiple-variables-separated-by-a-sla}{\url{tex.stackexchange}} post on that.}

\tikzsetnextfilename{simplepgfexample4}
\begin{figure*}[h]
	\centering
	\begin{tikzpicture}
	\def\mynumcurves{21}
	\begin{axis}[
		axis lines=center,
		width=\textwidth, height=\textwidth/\goldenratio,
		xlabel=$x$, ylabel={$y$},
		grid=both,
		ymax=1.1,
		ymin=-1.1,
		]
		\foreach [evaluate=\i as \mix using (\i-1)*100/(\mynumcurves)] \i in {1, 2,...,\mynumcurves}{
			\edef\temp{%  \edef\temp everything is expanded, except prefixed with \noexpand 
			\noexpand\addplot[
				domain=-3.2:3.2,
				samples=201, 
				line width=1+\mix/100*2,
				color=\colorforcurvesv!\mix!\colorforcurvesi]
				{exp(-x^2)*sin(deg((100-\mix)/15*x))};
			}\temp
		}
	\end{axis}
\end{tikzpicture}
	\caption{A family of modulated Gaussian curves.}
	\label{fig:simplepgfexample4}
\end{figure*}

\tikzsetnextfilename{simplepgfexample5}
\begin{figure*}[h]
	\centering
	\input{../src/figures/simplepgfexample5.tex}
	\caption{Now using opacity ...}
	\label{fig:simplepgfexample5}
\end{figure*}

\tikzsetnextfilename{simplepgfexample6}
\begin{figure*}[h]
	\centering
	\begin{tikzpicture}
	\def\mynumcurves{11}
	\begin{axis}[
		axis lines=center,
		width=\textwidth, height=\textwidth/\goldenratio,
		xlabel=$x$, ylabel={$y$},
%		grid=both,
		ymax=1.1,
		ymin=-1.1,
		legend style={draw=none,anchor=north east, at={(1,1)}},
		]
		%% generate legends (this is done first, otherwise entries will be overwritten)
		\pgfmathsetmacro{\halfinterval}{int(\mynumcurves/2)}
		\foreach [evaluate=\i as \mix using (\i-1)*100/(\mynumcurves)] \i in {1,\halfinterval,\mynumcurves}{
			\pgfmathsetmacro{\freq}{(100-\mix)/15}
			\edef\temp{%  \edef\temp everything is expanded, except prefixed with \noexpand 
				\noexpand
				\addlegendimage{
					line width=1+\mix/100*2,
					color=\colorforcurvesv!\mix!\colorforcurvesi,
					opacity=\mix/100*0.95+0.05,
				}
				\noexpand\ifthenelse
				{ % condition
					\i=\halfinterval
				}
				{ % then clause
					\noexpand\addlegendentry{$\vdots$}
				}
				{% else
					\noexpand\addlegendentry{$f_{\i}(x) = e^{-x^2}\sin(\noexpand\pgfmathprintnumber{\freq}\,x)$}
				}   				
			}\temp
		}
		%% generate the plots
		\foreach [evaluate=\i as \mix using (\i-1)*100/(\mynumcurves)] \i in {1, 2,...,\mynumcurves}{
			\edef\temp{%  \edef\temp everything is expanded, except prefixed with \noexpand 
				\noexpand\addplot[
				domain=-3.2:3.2,
				samples=201, 
				opacity=\mix/100*0.95+0.05,
				line width=1+\mix/100*2,
				color=\colorforcurvesv!\mix!\colorforcurvesi]
				{exp(-x^2)*sin(deg((100-\mix)/15*x))};
			}\temp
		}
	\end{axis}
\end{tikzpicture}
	\caption{... and a legend.}
	\label{fig:simplepgfexample6}
\end{figure*}



Regarding marks\index{axis!marks} and axis (and hand-crafted grids\index{axis!grid}) see \cref{fig:discretization}.

\tikzsetnextfilename{discretization}
\begin{figure*}
	\centering
	
\def\myplotxdist{8}
\def\myplotydist{7}
\begin{tikzpicture}
	\def\thefunction{(2.5 + x^8 + cos(deg(5*pi*x)) + cos(deg(6*pi*x)) + sin(deg(3*pi*x)))*exp(-x*x*2)*0.19}
	\def\myxmin{0}
	\def\myxmax{1}
	\def\myymin{0}
	\def\myymax{1}
	\def\mynsamples{30}
	\def\sampletimestep{(\myxmax - \myxmin)/(\mynsamples-1)}
	\def\mycolor{\colorforcurvesi}
	\def\mycolorii{\colorforcurvesii}
	\def\nquantsteps{30}
	\def\quantizeres{1/(\nquantsteps)}
	\def\mygridcolor{gray!20}
	\node (orig) at (0,0) {};
	\node[anchor=center] (conti) at (0,0) {
		\begin{tikzpicture}
			\begin{axis}[
				xmin=\myxmin,xmax=\myxmax,
				ymin=\myymin,ymax=\myymax,
				xlabel= $t  \in \realnumbers$,
				ylabel=$f(t) \in \realnumbers$,	
				xtick=\empty, 
				ytick=\empty, 
				]
				\addplot[domain=\myxmin:\myxmax,samples=1000, \mycolor]{\thefunction};
			\end{axis}
		\end{tikzpicture}
	};
	\node[anchor=center] (timedisc) at (\myplotxdist,0) {
		\begin{tikzpicture}
			\begin{axis} [
				xmin=\myxmin,xmax=\myxmax,
				ymin=\myymin,ymax=\myymax,
				xlabel={$n\sampletime, n \in \integers$},
				ylabel={$f(n\sampletime)  \in \realnumbers$},	
				xtick=\empty, 
				ytick=\empty, 
				]
				\foreach \i in {1, 2,...,\mynsamples}{
					\addplot[\mygridcolor, line width=0.1pt] coordinates {
						((\i-1)*\sampletimestep,\myymin) ((\i-1)*\sampletimestep,\myymax) 
					};
				}
				\addplot+[only marks, domain=\myxmin:\myxmax,samples=\mynsamples, mark color=\mycolor, mark=|, mark size=3pt, \mycolor]{\thefunction};
				\addplot+[only marks, domain=\myxmin:\myxmax,samples=\mynsamples, mark color=\mycolor, mark=o, mark size=1pt, mark options={fill=\mycolor, draw=\mycolor,}]{\thefunction};
				\addplot[domain=\myxmin:\myxmax,samples=1000, densely dotted, \mycolorii]{\thefunction};
			\end{axis}
		\end{tikzpicture}
	};	
	\node[anchor=center] (valdisc) at (0,-\myplotydist) {
		\begin{tikzpicture}
			\begin{axis} [
				xmin=\myxmin,xmax=\myxmax,
				ymin=\myymin,ymax=\myymax,
				xlabel= {$t  \in \realnumbers$},
				ylabel={$\lfloor f(t) \rfloor \in \{m\quantizestep \,\vert\, m \in \integers\}$},	
				xtick=\empty, 
				ytick=\empty, 
				]
				\foreach \i in {1, 2,...,\nquantsteps}{
					\addplot[\mygridcolor, line width=0.1pt] coordinates {
						(\myxmin, {(\i-1)*\quantizeres}) (\myxmax, {(\i-1)*\quantizeres}) 
					};
				}
				\addplot+[only marks,domain=\myxmin:\myxmax, samples=1001, \mycolor,mark=o,mark options={xscale=0.005, yscale=0.25}]{floor(\thefunction*\nquantsteps)/\nquantsteps};
				\addplot[domain=\myxmin:\myxmax,samples=1000, dotted, \mycolor]{\thefunction};
				\addplot[domain=\myxmin:\myxmax,samples=1000, densely dotted, \mycolorii]{\thefunction};
			\end{axis}
		\end{tikzpicture}
	};				
	\node[anchor=center] (timevaldisc) at (\myplotxdist,-\myplotydist) {
		\begin{tikzpicture}
			\begin{axis} [
				xmin=\myxmin,xmax=\myxmax,
				ymin=\myymin,ymax=\myymax,
				xlabel= {$n\sampletime, n \in \integers$},
				ylabel={$\lfloor f(n\sampletime) \rfloor  \in \{m\quantizestep\,\vert\, m \in \integers\}$},	
				xtick=\empty, 
				ytick=\empty, 
				]
				\foreach \i in {1, 2,...,\nquantsteps}{
					\addplot[\mygridcolor, line width=0.1pt] coordinates {
						(\myxmin, {(\i-1)*\quantizeres}) (\myxmax, {(\i-1)*\quantizeres}) 
					};
				}
				\foreach \i in {1, 2,...,\mynsamples}{
					\addplot[\mygridcolor, line width=0.1pt] coordinates {
						((\i-1)*\sampletimestep,\myymin) ((\i-1)*\sampletimestep,\myymax) 
					};
				}
				\addplot+[only marks, domain=\myxmin:\myxmax,samples=\mynsamples, mark color=\mycolor, mark=+, mark size=3pt, \mycolor] {floor(\thefunction*\nquantsteps)/\nquantsteps};
				\addplot+[only marks, domain=\myxmin:\myxmax,samples=\mynsamples, mark color=\mycolor, mark=o, mark size=1pt, mark options={fill=\mycolor, draw=\mycolor,}]{floor(\thefunction*\nquantsteps)/\nquantsteps};
				\addplot[domain=\myxmin:\myxmax,samples=1000, densely dotted, \mycolorii]{\thefunction};
			\end{axis}
		\end{tikzpicture}
	};				
\end{tikzpicture}
	\caption{A visual overview about all pairings of discretization in time (sampling) and value (quantization, by slight abuse of floor-notation).}\label{fig:discretization}
\end{figure*}


Moreover, we provide \cref{fig:sampling} simply because it fits into the context of \cref{fig:discretization} (see \cref{exa:sampling}).


\begin{myexample}[Aliasing]\label{exa:sampling}
	Let $f_0 = 0.125$. \cref{fig:sampling} shows three different signals
	\[
	\displaystyle f(t) = \cos(2\pi t f_0),\quad
	g(t) = \cos(2.5\pi t),\quad
	h(t) = \cos(1.75\pi t)
	\]
	sampled with sampling frequency \(\samplingfreq=1\), \ie \(\samplingfreq = 1 = {1}/\sampletime\). $g$ and $h$ correspond to $\samplingfreq \pm f_0$.
\end{myexample}

\tikzsetnextfilename{sampling}
\begin{figure*}
	\centering
	
\begin{tikzpicture}
	\def\myxmin{0}
	\def\myxmax{12}
	\def\myymin{-1.5}
	\def\sampfreq{1}
	\def\myymax{1.5}
	\def\mynsamples{12}
	\def\sampletimestep{1/\sampfreq}
	\def\mycolori{\colorforcurvesi}
	\def\mycolorii{\colorforcurvesv}
	\def\mycoloriii{\colorforcurvesii}
	\def\sigfreq{0.125}
	\def\thefunctionf{cos(deg((\sigfreq)*2*pi*x))}
	\def\thefunctiong{cos(deg((\sigfreq + 1*\sampfreq)*2*pi*x))}
	\def\thefunctionh{cos(deg((-\sigfreq + 1*\sampfreq)*2*pi*x))}
			\begin{axis}[
				width=\textwidth*0.8,
				height=5cm,
				xmin=\myxmin,xmax=\myxmax,
				ymin=\myymin,ymax=\myymax,	
				axis lines=center,	
				legend style={draw=none, at={(1.05,0.5)},anchor=west},
				xtick=\empty, ytick=\empty, 
				]
				\addplot[domain=\myxmin:\myxmax,samples=1000, \mycolori]{\thefunctionf};
				\addlegendentry{\(f(t)\)}
				\addplot[domain=\myxmin:\myxmax,samples=1000, \mycolorii]{\thefunctiong};
				\addlegendentry{\(g(t)\)}
				\addplot[domain=\myxmin:\myxmax,samples=1000, \mycoloriii]{\thefunctionh};
				\addlegendentry{\(h(t)\)}
				\addplot+[only marks, domain=\myxmin:\myxmax,samples=\mynsamples+1, mark color=\mycolori, mark=*, mark options={fill=\mycolori, draw=\mycolori,}]{\thefunctionf};
				\addlegendentry{\(f(n\sampletime)\)}
				\addplot+[only marks, domain=\myxmin:\myxmax,samples=\mynsamples+1, mark color=\mycolorii, mark=o, mark options={fill=\mycolorii, draw=\mycolorii,}]{\thefunctiong};
				\addlegendentry{\(g(n\sampletime)\)}
				\addplot+[only marks, domain=\myxmin:\myxmax,samples=\mynsamples+1, mark color=\mycoloriii, mark=-, mark options={fill=\mycoloriii, draw=\mycoloriii,}]{\thefunctiong};
				\addlegendentry{\(h(n\sampletime)\)}
				\foreach \i in {1, 2,...,\mynsamples}{
					\addplot[black!25, line width=0.1pt] coordinates {
						({(\i-1)*\sampletimestep},\myymin) ({(\i-1)*\sampletimestep},\myymax) 
					};
				}
			\end{axis}
\end{tikzpicture}
	\caption{\Cf \cref{exa:sampling}.}\label{fig:sampling}
\end{figure*}

			
	
	\cleardoublepage % move to new page
	\phantomsection % create a target for hyperref
	\addcontentsline{toc}{chapter}{Bibliography}
	\bibliographystyle{plainnat}
	\bibliography{../src/para/references}
	\cleardoublepage
	\printindex
\end{document}