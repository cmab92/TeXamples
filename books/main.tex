\documentclass[%
twoside,symmetric,	% main question: print-version or online-only?
nols, % suppress letterspacing for small caps text
a4paper, % ...
notoc,	% i want that custom
justified,	% unusual for tufte-book! Anyway ...
nobib, % custom references
]{tufte-book} % 

%% Change page tiling:
\geometry{
	left=24.8mm, % left margin
	textwidth=110mm, % orig: 100mm, main text block
	marginparsep=8.2mm, % orig: 8.2mm, margin between main text block and margin notes
	marginparwidth=44.4mm % orig: 49.4mm, width of margin notes
}

\usepackage{fontspec} % e.g. using different fonts
\usepackage{pgfplots} % :)
\usepackage{amsmath,amssymb,amsfonts,amsthm,mathrsfs,mathtools,xfrac} % Regarding equations
\usepackage[utf8]{luainputenc}	% LuaLaTex!
\usepackage[capitalize,nameinlink,noabbrev]{cleveref} % simplifies \ref usage and improves formatting
\usepackage[framemethod=tikz]{mdframed} % boxes (e.g. for theorems, definitions, etc.)
\usepackage{emoji} % just for fun
\usepackage{lipsum} % blind text
\usepackage{psvectorian} % just for fun
\usepackage{tabto}
\usepackage{dirtytalk}	% mainly \say command
\usepackage{ifthen} % ... e.g. for more complicated plots
\usepackage{romannum}	% ...
\usepackage{array,readarray}	% read arrays
\usepackage{siunitx}	% ...
\usepackage{epigraph}	% just for fun
\usepackage{fourier}	% libertine eulervm
\usepackage{verbatim} % for \verbatiminput


% Optional: make index entries smaller

%% Language
\usepackage[ngerman,english]{babel} % mainly for correct word-breaking (the last-mentioned is dominant, e.g. Contents instead of Inhaltsverzeichnis)

%% Regarding bibliography
\usepackage{natbib}
\usepackage{bibentry}
\nobibliography*

%% index (:
%% Links (include after natbib for use in bibliography)
\usepackage{hyperref} % references, citations, URLs, and table of contents clickable in PDF
\usepackage{url}
\usepackage{imakeidx}       % index
% Create index using makeindex
\makeindex[intoc, columns=2, title=Index] % an index with hyperlink is a little more complicated. Yet, Imo, this isnt need, as an index is not needed in a PDF (use ctrl+f)...

%% lettrine & fonts
%\usepackage{fontspec}
\usepackage{lettrine}

%% Regarding pgfplots and tikz
\usetikzlibrary{external}
\tikzexternalize[prefix=tikz/]
\tikzset{external/only named}

\usetikzlibrary{calc,arrows,arrows.meta,angles,quotes,plotmarks,fadings,shapes,spy,mindmap,backgrounds}
\usepgfplotslibrary{colorbrewer,fillbetween}
\pgfplotsset{compat=newest}

%% Regarding TOC
\setcounter{secnumdepth}{2} % activate section and chapter numbering
\setcounter{tocdepth}{1} % section and chapter number-depth in TOC

% colormaps
\usepgfplotslibrary{colorbrewer} % cf. https://tikz.dev/pgfplots/libs-colorbrewer
\def\mycolorscheme{RdYlBu}

\pgfplotsset{colormap/\mycolorscheme-11} 

\def\colorforcurvesi{\mycolorscheme-M}
\def\colorforcurvesii{\mycolorscheme-K}
\def\colorforcurvesiii{\mycolorscheme-I}
\def\colorforcurvesiv{\mycolorscheme-E}
\def\colorforcurvesv{\mycolorscheme-C}
\def\cfcA{\mycolorscheme-A}
\def\cfci{\mycolorscheme-B}
\def\cfcii{\mycolorscheme-C}
\def\cfcE{\mycolorscheme-E}
\def\cfciii{\mycolorscheme-F}
\def\cfciv{\mycolorscheme-G}
\def\cfcv{\mycolorscheme-H}
\def\cfcvi{\mycolorscheme-I}
\def\cfcvii{\mycolorscheme-J}
\def\cfcviii{\mycolorscheme-L}
\def\cfcix{\mycolorscheme-N}

\def\posmatcolor{\mycolorscheme-B}
\def\negmatcolor{\mycolorscheme-N}

%
\def\mylocalcolorproof{\mycolorscheme-I}
\def\mylocalcolordef{\mycolorscheme-J}
\def\mylocalcolortheo{\mycolorscheme-N}

\def\urlcolor{\mycolorscheme-O}
\def\linkcolor{\mycolorscheme-O}
\def\citecolor{\mycolorscheme-A}

\gdef\mysidenotetextcolor{\mycolorscheme-N}
\gdef\mycitationtextcolor{\mycolorscheme-B}
\gdef\mychaptertitlecolor{\mycolorscheme-O}

\hypersetup{
	linkcolor=\linkcolor,
	citecolor=\citecolor,
	urlcolor=\urlcolor,
	colorlinks=true,
}


\gdef\mymathcolor{black}
\everymath{\color{\mymathcolor}}       % inline math
\everydisplay{\color{\mymathcolor}}    % display math

\renewcommand\thefootnote{\sffamily\textcolor{\mysidenotetextcolor}{\arabic{footnote}}}

\setsidenotefont{
	\sffamily
	\footnotesize
	\color{\mysidenotetextcolor}
}   

\setcaptionfont{
	\sffamily
	\footnotesize
	\color{\mysidenotetextcolor}
}   

\setmarginnotefont{
	\sffamily
	\small
	\color{\mysidenotetextcolor}
}   

\setcitationfont{
	\sffamily
	\color{\mycitationtextcolor}
}

\gdef\mysidenotefont{\sffamily\textcolor{\mysidenotetextcolor}}

\def\sectionfont{\sffamily\LARGE}


% chapter format
\titleformat{\chapter}%
{\huge\sffamily\scshape\color{\mychaptertitlecolor}}% format applied to label+text
{{\thechapter}\llap}% label
{1em}% horizontal separation between label and title body
{}% before the title body
[]% after the title body

% section format
\gdef\mysectionfont{\sffamily\Large\color{\mychaptertitlecolor}}
\titleformat{\section}%
{\mysectionfont}% !!!!!!!!!! % \itshape % format applied to label+text
{{\thesection}\llap}% {\llap{\colorbox{white}{\parbox{1.5cm}{\hfill\color{white}\thesection}}}}% label
{1em}% horizontal separation between label and title body
{}% before the title body
[]% after the title body

% subsection format
\titleformat{\subsection}%
{\sffamily\large\itshape\color{\mychaptertitlecolor}}%  % 
{{\thesubsection}\llap}% {\llap{\colorbox{white}{\parbox{1.5cm}{\hfill\color{white}\thesubsection}}}}% label
{1em}% horizontal separation between label and title body
{}% before the title body
[]% after the title body


%% alternative fonts
\newfontface\kitschy{ZapfinoExtraLT-Four} % really fancy

\makeatletter
\newcommand\chapterauthor[1]{#1\gdef\@chapterauthor{#1}}
\def\@chapterauthor{}
\fancypagestyle{mystyle}{%
	\fancyhf{}%
	\renewcommand{\chaptermark}[1]{\markboth{##1}{}}%
	\fancyhead[LE]{\thepage\quad{\newlinetospace{\leftmark}}}% 
	\fancyhead[RO]{\smallcaps{\newlinetospace{\@chapterauthor}}\quad\thepage}%
}
\fancypagestyle{mysecondstyle}{% not needed here ...
	\fancyhf{}%
	\renewcommand{\chaptermark}[1]{\markboth{##1}{}}%
	\fancyhead[LE]{\thepage\quad{\newlinetospace{\leftmark}}}% 
	\fancyhead[RO]{\MakeUppercase{\newlinetospace{\@chapterauthor}}\quad\thepage}%
}
\makeatother


\usepackage{keyval}
\makeatletter

%%% Definition of keys
\define@key{reuse}{currentkind}{\def\@tempa{#1}}
\define@key{reuse}{currententry}{\def\reuse@current{#1}}
\def\define@reuse@key#1{\define@key{reuse}{#1}{%
		\global\@namedef{reuse@\reuse@current @#1}{##1}}}
\define@reuse@key{title}
\define@reuse@key{isbn}
\define@reuse@key{url}
\define@reuse@key{year}
\define@reuse@key{note}

%%% The macro substituted to @book, @article, etc.
\def\reuse@find#1#{%
	\lowercase{\setkeys{reuse}{currentkind=#1}}%
	\ifcsname reuse@type@\@tempa\endcsname
	\expandafter\@gobble
	\else
	\begingroup\makeatother
	\expandafter\reuse@extract
	\fi}

%%% the macro for extracting the fields
\def\reuse@extract#1{\setkeys{reuse}{currententry=#1}\endgroup}

%%% Ignore @preamble and @string
\let\reuse@type@preamble\@empty
\let\reuse@type@string\@empty

%%% Error management
\def\reuse@error#1#2{%
	\PackageError{reuse}
	{Undefined key `#1' or empty value for `#2'}
	{The key you used is wrong or the value to `#2' has not been set}}

%%% The four user level macros
\newcommand\newbibfield[1]{\define@reuse@key{#1}}

\def\usebibentry#1#2{\@ifundefined{reuse@#1@#2}
	{\reuse@error{#1}{#2}}
	{\@nameuse{reuse@#1@#2}}}

\newcommand{\usebibentryurl}[2][|]{\@ifundefined{reuse@#2@url}
	{\reuse@error{#2}{url}}
	{\@usebibentryurl{#1}{#2}}}

\@ifpackageloaded{hyperref}{\@tempswatrue}{\@tempswafalse}
\if@tempswa
\def\@usebibentryurl#1#2{%
	\scantokens{\url{\csname reuse@#2@url\endcsname}\endinput}}
\else
\def\@usebibentryurl#1#2{%
	\toks@=\expandafter\expandafter\expandafter
	{\csname reuse@#2@url\endcsname}%
	\scantokens\expandafter{%
		\expandafter\url\expandafter#1\the\toks@#1\endinput}}
\fi

\newcommand{\bibinput}[1]{%
	\begingroup
	\catcode`\%=12
	\let\KV@err=\@gobble
	\let\KV@errx=\@gobble
	\let\XKV@err=\@gobble
	\begingroup\lccode`\~=`\@
	\lowercase{\endgroup\let~}\reuse@find
	\catcode`\@=\active \input{#1.bib}\endgroup}

\makeatother 
\newcommand{\mycite}[1]{\cite{#1}}
\newcommand{\mycitepage}[2]{\cite{#1}, p. {#2}}

\newcommand{\mysay}[1]{\say{\textit{#1}}}
\newcommand{\citesourcename}[1]{\usebibentry{#1}{title}}
\newcommand{\citeauthorname}[1]{\citeauthor{#1}}
\newcommand{\citepublicationyear}[1]{{\usebibentry{#1}{year}}}
\newcommand{\citesay}[1]{\hfill \\ ---\,\citeauthor{#1}, \usebibentry{#1}{title}, \usebibentry{#1}{year}}

%% regarding epigraphs
\setlength\epigraphwidth{.75\textwidth}
\setlength\epigraphrule{0pt}

\let\originalepigraph\epigraph 
\renewcommand\epigraph[2]%
{\originalepigraph{\textit{\say{#1}}}{#2}}


\newenvironment{mytextnarrowcite}
{
	\vspace{2mm}\begin{adjustwidth}{3em}{3em}} % customize the left and right margins here
	{\end{adjustwidth}
	\vspace{2mm}}
	
\newcommand{\myepigraphintext}[3]{
	\vspace{2mm}
	\begin{mytextnarrowcite}\label{#1}
		{\small\textit{\say{#2}}} \small\citesay{#3}
	\end{mytextnarrowcite}
	\vspace{2mm}
}


\def\myltab{\tabto{4cm}} % glossary and nomenclature
\def\goldenratio{1.6180339887} % just for fun

\newcommand{\ie}{\textit{i.e.,}~}
\newcommand{\cf}{\textit{cf.}~}
\newcommand{\Cf}{\textit{Cf.}~}
\newcommand{\eg}{\textit{e.g.,}~}
\newcommand{\Eg}{\textit{E.g.,}~}
\newcommand{\sic}{[\textit{sic}]}
\newcommand{\wrt}{w.r.t.~}
% theorem

%\def\mylocalcolor{white}
\def\mylocalopacval{20}
\def\mylocalopacvalproof{10}
\def\mylocalopacvaldef{20}
\def\mylocalopacvaltheo{30}
\def\mylocaltopskip{1cm}
%%%%%%%%%%%%%%%%%%%%%%%%%%%%%%%%%%%%%%%%%%%%%%%%%%%%%%%%%%%%%%%%%%%%%%%%%%%%%%%%%%%%%%%%%%%%%%%%%%%%%
\makeatletter
\DeclareDocumentCommand{\mdtheorem}{ O{} m o m o }%
{\ifcsdef{#2}%
	{\mdf@PackageWarning{Environment #2 already exits\MessageBreak}}%
	{%
		\IfNoValueTF {#3}%
		{%#3 not given -- number relationship
			\IfNoValueTF {#5}%
			{%#3+#5 not given
				\@definecounter{#2}%
				\expandafter\xdef\csname the#2\endcsname{\@thmcounter{#2}}%
				\newenvironment{#2}[1][]{%
					\refstepcounter{#2}%
					\ifstrempty{##1}%
					{\let\@temptitle\relax}%
					{%
						\def\@temptitle{\mdf@theoremseparator%
							\mdf@theoremspace%
							\mdf@theoremtitlefont%
							##1}%
						\mdf@thm@caption{#2}{{#4}{\csname the#2\endcsname}{##1}}%
					}%
					\begin{mdframed}[#1,frametitle={\strut#4\ \csname the#2\endcsname%
							\@temptitle}]}%
					{\end{mdframed}}%
				\newenvironment{#2*}[1][]{%
					\ifstrempty{##1}{\let\@temptitle\relax}{\def\@temptitle{\mdf@theoremseparator\ ##1}}% <- the problem was here
					\begin{mdframed}[#1,frametitle={\strut#4\@temptitle}]}%
					{\end{mdframed}}%
			}%
			{%#5 given -- reset counter
				\@definecounter{#2}\@newctr{#2}[#5]%
				\expandafter\xdef\csname the#2\endcsname{\@thmcounter{#2}}%
				\expandafter\xdef\csname the#2\endcsname{%
					\expandafter\noexpand\csname the#5\endcsname \@thmcountersep%
					\@thmcounter{#2}}%
				\newenvironment{#2}[1][]{%
					\refstepcounter{#2}%
					\ifstrempty{##1}%
					{\let\@temptitle\relax}%
					{%
						\def\@temptitle{\mdf@theoremseparator%
							\mdf@theoremspace%
							\mdf@theoremtitlefont%
							##1}%
						\mdf@thm@caption{#2}{{#4}{\csname the#2\endcsname}{##1}}%
					}
					\begin{mdframed}[#1,frametitle={\strut#4\ \csname the#2\endcsname%
							\@temptitle}]}%
					{\end{mdframed}}%
				\newenvironment{#2*}[1][]{%
					\ifstrempty{##1}%
					{\let\@temptitle\relax}%
					{%
						\def\@temptitle{\mdf@theoremseparator%
							\mdf@theoremspace%
							\mdf@theoremtitlefont%
							##1}%
						\mdf@thm@caption{#2}{{#4}{\csname the#2\endcsname}{##1}}%
					}%
					\begin{mdframed}[#1,frametitle={\strut#4\@temptitle}]}%
					{\end{mdframed}}%
			}%
		}%
		{%#3 given -- number relationship
			\global\@namedef{the#2}{\@nameuse{the#3}}%
			\newenvironment{#2}[1][]{%
				\refstepcounter{#3}%
				\ifstrempty{##1}%
				{\let\@temptitle\relax}%
				{%
					\def\@temptitle{\mdf@theoremseparator%
						\mdf@theoremspace%
						\mdf@theoremtitlefont%
						##1}%
					\mdf@thm@caption{#2}{{#4}{\csname the#2\endcsname}{##1}}%
				}
				\begin{mdframed}[#1,frametitle={\strut#4\ \csname the#2\endcsname%
						\@temptitle}]}%
				{\end{mdframed}}%
			\newenvironment{#2*}[1][]{%
				\ifstrempty{##1}{\let\@temptitle\relax}{\def\@temptitle{:\ ##1}}%
				\begin{mdframed}[#1,frametitle={\strut#4\@temptitle}]}%
				{\end{mdframed}}%
		}%
	}%
}
\makeatother
%%%%%%%%%%%%%%%%%%%%%%%%%%%%%%%%%%%%%%%%%%%%%%%%%%%%%%%%%%%%%%%%%%%%%%%%%%%%%%%%%%%%%%%%%%%%%%%%%%%%%%%%%%%%%%%%%%%%%%%%%%%%%%%%%%%%%%%%%%%%%%%%%
\newenvironment{myoutline}
{\mdfsetup{
		frametitle={},
		%		innertopmargin=10pt,
		%		frametitleaboveskip=-\ht\strutbox,
		%		frametitlealignment=\center
	}
	\begin{mdframed}%
	}
	{\end{mdframed}}
%%%%%%%%%%%%%%%%%%%%%%%%%%%
\mdfdefinestyle{myStyle}{%
	backgroundcolor=\mylocalcolor!\mylocalopacval, 
	nobreak=true,
	linewidth=0pt,
	roundcorner=5pt,
	innertopmargin=-0.5ex,
}
\mdfdefinestyle{myproofStyle}{%
	%	backgroundcolor=\mylocalcolorproof!\mylocalopacvalproof, 
	nobreak=false,
	linewidth=0pt,
	roundcorner=5pt,
	innertopmargin=-0.5ex,
	theoremseparator={ ---},
	theoremspace=\space
}
\mdfdefinestyle{mydefStyle}{%
	backgroundcolor=\mylocalcolordef!\mylocalopacvaldef, 
	nobreak=true,
	linewidth=0pt,
	roundcorner=5pt,
	innertopmargin=-0.5ex,
}
\mdfdefinestyle{mytheoStyle}{%
	backgroundcolor=\mylocalcolortheo!\mylocalopacvaltheo, 
	nobreak=true,
	linewidth=0pt,
	roundcorner=5pt,
	innertopmargin=-0.5ex,
}

\mdtheorem[style=mydefStyle]{mydef}{Definition}
\numberwithin{mydef}{chapter}

\mdtheorem[style=myproofStyle]{myproof}{Proof}
\numberwithin{myproof}{chapter}

\mdtheorem[style=mytheoStyle]{mytheo}{Theorem}
\numberwithin{mytheo}{chapter}

\makeatletter
\if@cref@capitalise
\crefname{mydef}{Definition}{Definition}
\crefname{mytheo}{Theorem}{Theorem}
\crefname{myproof}{Proof}{Proof}
\else
\crefname{mydef}{definition}{definition}
\crefname{mytheo}{theorem}{theorem}
\crefname{myproof}{proof}{proof}
\fi
\makeatother

%

% proofs
\newtheorem{myshortproof*}{Proof}
\numberwithin{myshortproof*}{chapter} % important bit
% examples
\newtheorem{myexample}{Example}
\numberwithin{myexample}{chapter} % important bit
% lemma
\newtheorem{mylemma}{Lemma}
\numberwithin{mylemma}{chapter} % important bit


% title
\newcommand{\titleofthesis}{Der lustige Titel meiner Publikation}
\newcommand{\meinetolleuni}{Universität Hinterdupfing}
\newsavebox{\hslogo}
\savebox{\hslogo}[230mm]{
	\begin{tikzpicture}[overlay]
		\def\widthornament{4}
		\node (orig) at (1,7) {};
		\node[anchor=center] (2) at (orig) {
			\begin{pspicture}(0,0)(\widthornament,\widthornament)% 
				% pstricks in pgfplots
				% (x0,x1)(y0,y1), where x and y are lower left and upper right corner 
				\rput[tl](0,0){\psvectorian[width=\widthornament cm]{91}} % https://ftp.rrze.uni-erlangen.de/ctan/graphics/pstricks/contrib/pst-vectorian/doc/psvectorian.pdf
			\end{pspicture}
		};
		\node at ($(2.south) + (0.6,-2.7)$) {\small\kitschy \meinetolleuni};
	\end{tikzpicture}
}

\title[\titleofthesis]{%
	\usebox{\hslogo}
	\setlength{\parindent}{0pt}%
	\vspace{2cm} \par \Huge Der lustige Titel\\ meiner Publikation \par \vspace{0.5cm} 
	\small{
		Dissertation zur Erlangung des Doktorgrades	\textit{Dr. rer. tex.}\\
		der Fakultät für Irgendwas und Noch Etwas\\
		der \meinetolleuni\\
		\vspace{1cm} \\
		\textit{von} Irgend R. Jemanden aus Hinterdupfing $\quad$ | $\quad$ 2025 \par
		\textit{\meinetolleuni,} \textit{Institut für Form ohne Inhalt}
	}
}


\begin{document}
	% Frontmatter
	\maketitle% this prints the handout title, author, and date
% Second page
\setlength{\parindent}{0pt}%
\thispagestyle{empty}

This document was typeset in \LaTeX~using the \texttt{tufte-book} document class. All packages being used can be found in the corresponding  \href{https://github.com/cmab92/TeXamples/tree/master/books/main.tex}{git-repository}. \\\newline 

This document is licensed under the 
\href{https://creativecommons.org/licenses/by/4.0/}{Creative Commons Attribution 4.0 International License (CC BY 4.0)}. \\\newline 

\vfill
Christopher Bonenberger (2025) \titleofthesis \clearpage
% Third page
\setlength{\parindent}{0pt}%
\thispagestyle{empty}
\vfill

%%%%%%%%%%%%%%%%%%%%%%%%%%%%%%%%%%%%%%%%%%%%%%%%%%%%%%%%%%%%%%%%%%%%%%%%%%%%%%%%%%%%%%%%%%%%%%%%%%%

%\section*{Danksagung}
\thispagestyle{empty}
\setlength{\parindent}{0pt}%
\hfill %  Dedicated to ...
\begin{fullwidth}
		{\centering
			\kitschytwo\LARGE Gott sei Dank ...\par
		}
\end{fullwidth}
\setlength{\parindent}{0pt}%
\cleardoublepage


%%%%%%%%%%%%%%%%%%%%%%%%%%%%%%%%%%%%%%%%%%%%%%%%%%%%%%%%%%%%%%%%%%%%%%%%%%%%%%%%%%%%%%%%%%%%%%%%%%%

\section*{Abstract}
\thispagestyle{empty}
\setlength{\parindent}{0pt}%
\lipsum[1-2]  % man soll ja Redundanzen vermeiden ...

\cleardoublepage
	% Nomenclature & Glossary
	\section*{Nomenclature}\label{nomenclature}
\thispagestyle{empty}


% As a start:
\def\mylhypertarget{rkhsnomen}
\newcommand{\hilbertspace}{\hypersetup{linkcolor=\mymathcolor}\hyperlink{\mylhypertarget}{\mathcal{H}}}%
{$\hilbertspace$ \myltab \hypertarget{\mylhypertarget}{Hilbert space of functions} \par}

\def\mylhypertarget{eulernomen}
\newcommand{\euler}{\hypersetup{linkcolor=\mymathcolor}\hyperlink{eulernomen}{\text{e}}}%
{$\euler$ \myltab \hypertarget{\mylhypertarget}{Euler's number (mathematical constant)} \par}

\def\mylhypertarget{complexnrnomen}
\newcommand{\complexunit}{\hypersetup{linkcolor=\mymathcolor}\hyperlink{\mylhypertarget}{\mathfrak{i}}}%
{$\complexunit$ \myltab \hypertarget{\mylhypertarget}{Complex unit ($\complexunit^2=-1$)} \par}

% an easier version would be:
%\newcommand{\hilbertspace}{\mathcal{H}%
%{$\hilbertspace$ \tab Hilbert space of functions \par}

\newcommand{\myendofproof}{\hfill\text{${\Box} $}}
\newcommand{\myendofproofequation}{\hfill\text{$\tag*{\Box} $}}

\cleardoublepage
	
\section*{Glossary}\label{glossary}

\thispagestyle{empty}

\def\mylhypertarget{rkhsgloss}
\newcommand{\RKHS}{\hyperlink{\mylhypertarget}{RKHS}}
\newcommand{\RKHSlong}{Reproducing Kernel Hilbert Space}
{\RKHS \tab \hypertarget{\mylhypertarget}{\RKHSlong} \par}

\def\mylhypertarget{tocgloss}
\newcommand{\mytoc}{\hyperlink{\mylhypertarget}{TOC}}
\newcommand{\mytoclong}{Table of Contents}
{\mytoc \tab \hypertarget{\mylhypertarget}{\mytoclong} \par}













\cleardoublepage
	
	
	%% TOC
	\sffamily % sans serif
	\tableofcontents
	\normalfont % back to normal
	\cleardoublepage
	
	% Document body
	\bibinput{../src/para/references}
	\pagestyle{mystyle}	
	\setcounter{page}{1}
	\pagenumbering{arabic}
	
	\chapter{Introduction}\label{sec:introduction}
		\epigraph{I would prefer not to [...]}{\citesay{melville1980bartleby}}

Without some general background and an introduction to the basic concepts, such a course can hardly get started. And indeed, it makes sense to first look at the tools we’ll be using here in a broader context. So, let's go \emoji{nerd-face}

\newthought{In medias res.} \mypgf\footnote{\url{https://ctan.org/pkg/pgf}}~ stands for \mypgflong~ and its a low-level graphics engine providing drawing primitives. TikZ\footnote{\url{https://pgf-tikz.github.io/pgf/pgfmanual.pdf}} is a high-level interface to PGF, which allows more user-friendly commands. Tikz on its own is already a really powerful tool. However, on top of TikZ, there is another abstraction specifically for \emph{scientific plotting}: PGFPlots. Before we face the over 500-pages \href{https://pgfplots.sourceforge.net/pgfplots.pdf}{documentation} let’s first pay a little tribute to the developer, Christian Feuersänger! He has also published other TeX-related material, such as a concise \href{https://ins.uni-bonn.de/media/public/staff/feuersaenger/MeineKurzReferenz.pdf}{overview} of important commands.

		
	\chapter{TikZ and PGFplots\,--\,Basics}
		\section{The axis-environment}
			An natural point to start using pgfplots is the axis-environment\index{axis}. We begin with the following simple example, which generates \cref{fig:simplepgfexample1}.

\verbatiminput{../src/figures/simplepgfexample1.tex}

\begin{figure}[h]
	\centering
	\begin{tikzpicture}
	\begin{axis}[
		xlabel=$x$, ylabel={$f(x)$},
		]
		\addplot {exp(-x^2)*sin(deg(x))};
	\end{axis}
\end{tikzpicture}
	\caption{This is fine for a start.}
	\label{fig:simplepgfexample1}
\end{figure}
Of course, we still want to improve this in many ways. We begin with \emph{global} versus \emph{local} styles.

\newthought{Global styles} are default in the sense that they apply to all axes and plots in the document (unless overridden). Thus, the following 
\begin{verbatim}
	\pgfplotsset{
		every axis/.style = {
			axis lines=center,
			width=\textwidth, height=\textwidth/\goldenratio,
			grid=both,
		},
		every axis plot/.style = {
    		ultra thick, mark=*
		}
	}
\end{verbatim}

\pgfplotsset{
	every axis/.style = {
		axis lines=center,
		width=\textwidth, height=\textwidth/\goldenratio,
		grid=both,
	},
	every axis plot/.style = {
		ultra thick, mark=*
	}
}

makes \cref{fig:simplepgfexample1} looking like \cref{fig:simplepgfexample1withdefault}.

\begin{figure}[h]
	\centering
	\begin{tikzpicture}
	\begin{axis}[
		xlabel=$x$, ylabel={$f(x)$},
		]
		\addplot {exp(-x^2)*sin(deg(x))};
	\end{axis}
\end{tikzpicture}
	\caption{Same as \cref{fig:simplepgfexample1} but with global styles.}
	\label{fig:simplepgfexample1withdefault}
\end{figure}

\newthought{Cycle lists} can be pretty handy:

\begin{verbatim}
	\pgfplotsset{
		cycle list={
			{\colorforcurvesi, thick},
			{\colorforcurvesii, thick},
			{\colorforcurvesiii, thick},
			{\colorforcurvesiv, thick},
			{\colorforcurvesv, thick},
			{\colorforcurvesi, dashed},
			{\colorforcurvesii, dashed},
			{\colorforcurvesiii, dashed},
			{\colorforcurvesiv, dashed},
			{\colorforcurvesv, dashed},
		}
	}
\end{verbatim}

\begin{figure}[h]
	\centering
	\begin{tikzpicture}
	\begin{axis}[
		xlabel=$x$, ylabel={$f(x)$},
		]
		\addplot {exp(-x^2)*sin(deg(x))};
	\end{axis}
\end{tikzpicture}
	\caption{Same as \cref{fig:simplepgfexample1cycle} with more graphs to show cycle list.}
	\label{fig:simplepgfexample1withdefault}
\end{figure}

\pgfplotsset{
	cycle list={
		{\colorforcurvesi, thick},
		{\colorforcurvesii, thick},
		{\colorforcurvesiii, thick},
		{\colorforcurvesiv, thick},
		{\colorforcurvesv, thick},
		{\colorforcurvesi, dashed},
		{\colorforcurvesii, dashed},
		{\colorforcurvesiii, dashed},
		{\colorforcurvesiv, dashed},
		{\colorforcurvesv, dashed},
	}
}



Local styles are applied inside a specific axis or \verb*|\addplot| and override global settings.

%\verbatiminput{../src/figures/simplepgfexample2.tex}

\tikzsetnextfilename{simplepgfexample2}
\begin{figure}[h]
	\centering
	\begin{tikzpicture}
	\begin{axis}[
		axis lines=center,
		width=\textwidth, height=\textwidth/\goldenratio,
		xlabel=$x$, ylabel={$f(x)$},
		grid=both,
		ymax=1.1,
		ymin=-1.1,
		]
		\addplot[
			domain=-3.2:3.2,
			samples=101, 
			line width=3,
			\colorforcurvesi]
			{exp(-x^2)*sin(deg(4*x))};
	\end{axis}
\end{tikzpicture}
	\caption{This one is a little more sophisticated.}
	\label{fig:simplepgfexample2}
\end{figure}

Now lets go for several functions:

%\verbatiminput{../src/figures/simplepgfexample3.tex}

\tikzsetnextfilename{simplepgfexample3}
\begin{figure*}[h]
	\centering
	\begin{tikzpicture}
	\begin{axis}[
		axis lines=center,
		width=\textwidth, height=\textwidth/\goldenratio,
		xlabel=$x$, ylabel={$y$},
		grid=both,
		ymax=1.4,
		ymin=-1,
		ytickmin=-0.5,
		ytickmax=1,
		legend style={draw=none,anchor=north east, at={(1,1)}},
		]
		\addplot[
		domain=-3.2:3.2,
		samples=101, 
		line width=3,
		\colorforcurvesv]
		{exp(-(x+1)^2)*sin(deg(3*(x+1)))};
		\addlegendentry{$g(x) = \euler^{-(x+1)^2}\sin\left(3(x+1)\right)$}
		\addplot[
		domain=-3.2:3.2,
		samples=101, 
		line width=3,
		\colorforcurvesi]
		{exp(-(x-1)^2)*sin(deg(4*(x-1)))};
		\addlegendentry{$f(x) = \euler^{-(x-1)^2}\sin\left(4(x-1)\right)$}
	\end{axis}
\end{tikzpicture}
	\caption{This one is a little more sophisticated.}
	\label{fig:simplepgfexample3}
\end{figure*}


	\chapter[Miscellaneous]{Miscellaneous}
		We begin with some technicalities. 
		\cleardoublepage
		\section[References]{References of All Kinds}
			The basic structures for citations, references, and the like are defined in the following files:
\begin{itemize}
	\item \texttt{settings/colorsettings.tex} \\ 
	Among other things, this file defines the colors for all the different types of references. For example, this reference to \cref{sec:introduction} has the color \linebreak \verb|\linkcolor| (\cf \cref{fig:linkcolor}) due to:
	\begin{marginfigure}[0cm]
		\centering
		\begin{tikzpicture}
			\draw[fill=\linkcolor] (0, 0) rectangle ++(\marginparwidth,\marginparwidth/\goldenratio); 
		\end{tikzpicture}\caption{The color \texttt{\linkcolor}.}\label{fig:linkcolor}
	\end{marginfigure}
	\begin{verbatim}
		\def\mycolorscheme{RdYlBu}
		...
		\def\linkcolor{\mycolorscheme-O}
		...
		\hypersetup{
			linkcolor=\linkcolor,
			citecolor=\citecolor,
			urlcolor=\urlcolor,
			colorlinks=true,
		}
	\end{verbatim}
	\item \texttt{settings/usebibentry.tex} \\ 
	A relatively messy way to provide a programmable way to pull BibTeX info into the text (not necessarily as a formal reference).
	
	\item \texttt{../src/para/citecommands.tex} \\ 
	Some more or less useful commands for citations and different kinds of quotes.
	
	\item \texttt{../src/para/nomenclature.tex} \\ 
	In the simplest case, one would use such a file just to throw in all the \verb|\newcommand| for formula symbols, and the like. In this example, however, we go a bit further and equip everything with the corresponding hyperlinks and also generate a nomenclature. This allows the following\sidenote{Margin notes are also references and are assigned certain properties (font, color, \dots) in the aforementioned files. Just like references to equations or similar items. For example, a reference to the most beautiful equation in mathematics: \cref{eq:mostbeautifulequation}. Is it?}
	\begin{equation}\label{eq:mostbeautifulequation}
		\euler^{\complexunit\pi} - 1 = 0.
	\end{equation}
	Note that the symbols are linked to the nomenclature. Everything in the nomenclature is hand-crafted, although there are certainly various ready-made solutions available.
	
	\item \texttt{../src/para/glossary.tex} \\ 
	Pretty much the same as for the nomenclature. Simply lets try by referencing to an \RKHS. Of course, we may also set the \verb|\hypertarget| to somewhere else (\eg to a corresponding mathematical definition).
	
	\item \texttt{../src/para/references} \\ 
	Here, we add (literature) sources. Since the resulting section does not automatically appear in the \mytoc~ we need
	\begin{verbatim}
		\phantomsection
		\addcontentsline{toc}{chapter}{Bibliography}
	\end{verbatim}
	followed by
	\begin{verbatim}
		\bibliographystyle{plainnat}
		\bibliography{../src/para/references}
	\end{verbatim}
	Notably, \emph{before} using any literature-reference use:\\
	\verb|\bibinput{../src/para/references}|
	
\end{itemize}

So far, so good. Let’s turn to the next topic: citations.
			\cleardoublepage
		\section[Fonts]{Digression}
			\epigraph{Der Heide, für den dieser treffliche Turnierhelm geschmiedet wurde, muss einen kapitalen Kopf gehabt haben. Doch das Ärgste ist, dass ihm die Hälfte fehlt.}{\citesay{cervantes1605}}

\lettrine[lines=3, loversize=0.1]{\textcolor{black}{\Examplefontii O}}{f course}, there is no need for such artistic typesetting digressions. Yet, one must admit that it is rather pretty to look at. Moreover, it can be argued that in our attention-driven society, even in science communication, one should no longer hesitate to resort to somewhat unorthodox methods. Or precisely (in the present case) to methods that appear \emph{very} orthodox.
\\

\lettrine[lines=3, loversize=0.1]{\textcolor{\colorforcurvesi}{\Examplefonti H}}{owever}, this kind of procrastination requires some nice or at least striking fonts to be installed \emph{and} to be chosen. The latter is a rabbit hole on its own.\footnote{\Eg \href{https://www.fontspace.com/}{fontspace.com}. And: lo and behold! A font does not necessarily have to readable in order to catch some attention.
\vspace{5mm}

\scalebox{1.5}{\LARGE \Examplefontiii A B C E F }
\vspace{2mm}

\scalebox{1.5}{\LARGE \Examplefontiii G H I J K}
\vspace{2mm}

\scalebox{1.5}{\LARGE \Examplefontiii L M N O P}

} Under Linux you may add fonts system-wide by copying the corresponding \verb|.ttf|-file to the 

{\centering \verb|/usr/share/fonts/truetype/|\par}

or via font-manager.

Here the following fonts were used:
\begin{itemize}
	\item[\Large \Examplefontiii W] \href{https://www.fontspace.com/tosca-font-f10033}{Tosca-8Rln}
	\item[\Large \Examplefontiii X] \href{https://www.fontspace.com/medievalalphabet-font-f13634}{Medievalalphabet-4EY6}
	\item[\Large \Examplefontiii Y] \href{https://www.fontspace.com/spanish-army-shields-two-font-f16698}{SpanishArmyShieldsTwo-g9aE}
	\item[\Large \Examplefontiii Z] \href{https://www.fontspace.com/hansschoenspergerrandomish-font-f8903}{Hansschoenspergerrandomish-VW6B}.
\end{itemize}

{\reasonablefont The font \say{Garamond}, however, is a very down-to-earth alternative. Creative, but professional. For document-wide use simply go for 
	
	{\centering \verb|\usepackage{ebgaramond}|.\par}

}
Math-fonts are a topic for another time. Obvious choices are 

{\centering \verb|eulervm|, \verb|mathpazo|, and \verb|fourier|.\par} 

\lettrine[lines=3, loversize=0.1]{\textcolor{black}{\Examplefontiv W}}{e} conclude this digression with another pearl of \href{https://www.ziereis-faksimiles.de/}{Initialkunst} and the last of the above four example fonts.

			\clearpage
		\section[The Cost of Ink?]{The Cost of Ink?}
			\epigraph{A designer knows he has achieved perfection not when there is nothing left to add, but when there is nothing left to take away.}{\citesay{de1992wind}}

I haven’t actually read the book \say{\citesourcename{de1992wind}} by \citeauthorname{de1992wind}. Instead the above quote is taken from the \href{https://tug.ctan.org/macros/latex/contrib/tufte-latex/sample-book.pdf}{Tufte-LaTeX sample book}. The quote basically says that good design is reached through simplicity and clarity, not by adding more features or decoration. A design is complete when everything unnecessary has been removed, leaving only what is essential and functional. Although this seems to be a nice guideline regarding the \say{design} of scientific texts and figures, there are other opinions on the topic of design in the broadest sense. One particularly striking example:
\myepigraphintext{quote:hundertwasser}{Die gerade Linie ist ein wahres Werkzeug des Teufels. Wer sich ihrer bedient, hilft mit am Untergang der Menschheit.}{hundertwasserweb2}
In fact, \citeauthorname{hundertwasserweb1} is a stark contrast to \say{modern} conceptions of design. It is not only that he dislikes the straight line a lot\footnote{\Cf \say{\citesourcename{hundertwasserweb1}} (\cite{hundertwasserweb1})} a lot, but his conception of design is generally opposing the above. His paintings and especially his texts provoke, yet at times they also display a great sense of humor (see for example \cite{hundertwasserweb3}). At first glance, these statements, which seem almost anarchistic, do not appear to fit the given context. The connection, however, is made by a contemporary from the same time and country.
	\chapter[Anything Goes]{Anything Goes}
		\epigraph{Die wissenschaftliche Ausbildung [...] simplifiziert die \frqq Wissenschaft\flqq, indem sie die Akteure simplifiziert.}{\citesay{feyerabend1986wider}}

We don’t want to be accused of that! Instead, let’s take this statement as an excuse to dive into the \say{fun} of \verb|pgfplots|.
			
	
	\cleardoublepage % move to new page
	\phantomsection % create a target for hyperref
	\addcontentsline{toc}{chapter}{Bibliography}
	\bibliographystyle{plainnat}
	\bibliography{../src/para/references}
	\cleardoublepage
	\printindex
\end{document}