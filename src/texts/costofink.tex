\epigraph{A designer knows he has achieved perfection not when there is nothing left to add, but when there is nothing left to take away.}{\citesay{de1992wind}}

The following text is mainly an example regarding citation\index{citation}. I haven’t actually read the book \say{\citesourcename{de1992wind}} by \citeauthorname{de1992wind}. Instead the above quote is taken from the \href{https://tug.ctan.org/macros/latex/contrib/tufte-latex/sample-book.pdf}{Tufte-LaTeX sample book}. The quote basically says that good design is reached through simplicity and clarity, not by adding more features or decoration. A design is complete when everything unnecessary has been removed, leaving only what is essential and functional. A similar understanding of design is promoted by Steve Jobs, saying that \mysay{[Design] is not just what it looks like and feels like. Design is how it works.} (\cite{jobs2003design}). Although this seems to be a nice guideline regarding the \say{design} of scientific texts and figures, there are other opinions on the topic of design in the broadest sense. One particularly striking example:
\myepigraphintext{quote:hundertwasser}{Die gerade Linie ist ein wahres Werkzeug des Teufels. Wer sich ihrer bedient, hilft mit am Untergang der Menschheit.}{hundertwasserweb2}
In fact, \citeauthorname{hundertwasserweb1} is a stark contrast to \say{modern} conceptions of design. It is not only that he dislikes straight lines a lot\footnote{\Cf \say{\citesourcename{hundertwasserweb1}} (\cite{hundertwasserweb1}).} a lot, but his conception of design (or rather arts and architecture) is generally somewhat opposing the one that is expressed by the introductory quote. His paintings and especially his texts provoke, yet at times they also display a great sense of humor (see for example \cite{hundertwasserweb3}). At first glance, these statements, which seem almost anarchistic, do not appear to fit the given context. The connection, however, is made by a contemporary from the same time and country...