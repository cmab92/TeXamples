The basic structures for citations\index{citation}, references\index{references}, and the like are defined in the following files:
\begin{itemize}
	\item \texttt{settings/colorsettings.tex} \\ 
	Among other things, this file defines the colors for all the different types of references. For example, this reference to \cref{sec:introduction} has the color \linebreak \verb|\linkcolor| (\cf \cref{fig:linkcolor}) due to:
	\begin{marginfigure}[0cm]
		\centering
		\begin{tikzpicture}
			\draw[fill=\linkcolor] (0, 0) rectangle ++(\marginparwidth,\marginparwidth/\goldenratio); 
		\end{tikzpicture}\caption{The color \texttt{\linkcolor}.}\label{fig:linkcolor}
	\end{marginfigure}
	\begin{verbatim}
		\def\mycolorscheme{RdYlBu}
		...
		\def\linkcolor{\mycolorscheme-O}
		...
		\hypersetup{
			linkcolor=\linkcolor,
			citecolor=\citecolor,
			urlcolor=\urlcolor,
			colorlinks=true,
		}
	\end{verbatim}
	\item \texttt{settings/usebibentry.tex} \\ 
	A relatively messy way to provide a programmable way to pull BibTeX info into the text (not necessarily as a formal reference).
	
	\item \texttt{../src/para/citecommands.tex} \\ 
	Some more or less useful commands for citations and different kinds of quotes.
	
	\item \texttt{../src/para/nomenclature.tex} \\ 
	In the simplest case, one would use such a file just to throw in all the \verb|\newcommand| for formula symbols, and the like. In this example, however, we go a bit further and equip everything with the corresponding hyperlinks and also generate a nomenclature. This allows the following\sidenote{Margin notes are also references and are assigned certain properties (font, color, \dots) in the aforementioned files. Just like references to equations or similar items. For example, a reference to the most beautiful equation in mathematics: \cref{eq:mostbeautifulequation}. Is it?}
	\begin{equation}\label{eq:mostbeautifulequation}
		\euler^{\complexunit\pi} - 1 = 0.
	\end{equation}
	Note that the symbols are linked to the nomenclature. Everything in the nomenclature is hand-crafted, although there are certainly various ready-made solutions available.
	
	\item \texttt{../src/para/glossary.tex} \\ 
	Pretty much the same as for the nomenclature. Simply lets try by referencing to an \RKHS. Of course, we may also set the \verb|\hypertarget| to somewhere else (\eg to a corresponding mathematical definition).
	
	\item \texttt{../src/para/references} \\ 
	Here, we add (literature) sources. Since the resulting section does not automatically appear in the \mytoc~ we need
	\begin{verbatim}
		\phantomsection
		\addcontentsline{toc}{chapter}{Bibliography}
	\end{verbatim}
	followed by
	\begin{verbatim}
		\bibliographystyle{plainnat}
		\bibliography{../src/para/references}
	\end{verbatim}
	Notably, \emph{before} using any literature-reference use\\
	\verb|\bibinput{../src/para/references}|\\
	within the \verb|document|.
	
\end{itemize}

So far, so good. Let’s turn to the next topic: citations. 