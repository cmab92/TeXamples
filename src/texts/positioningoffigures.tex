\epigraph{Der Heide, für den dieser treffliche Turnierhelm geschmiedet wurde, muss einen kapitalen Kopf gehabt haben. Doch das Ärgste ist, dass ihm die Hälfte fehlt.}{\citesay{cervantes1605}}

\lettrine[lines=3, loversize=0.1]{\textcolor{black}{\Examplefontii O}}{f course}, there is no need for such artistic typesetting digressions. Yet, one must admit that it is rather pretty to look at. Moreover, it can be argued that in our attention-driven society, even in science communication, one should no longer hesitate to resort to somewhat unorthodox methods. Or precisely (in the present case) to methods that appear \emph{very} orthodox.
\\

\lettrine[lines=3, loversize=0.1]{\textcolor{\colorforcurvesi}{\Examplefonti H}}{owever}, this kind of procrastination requires some nice or at least striking fonts to be installed \emph{and} to be chosen. The latter is a rabbit hole on its own.\footnote{\Eg \href{https://www.fontspace.com/}{fontspace.com}. And: lo and behold! A font does not necessarily have to readable in order to catch some attention.
\vspace{5mm}

\scalebox{1.5}{\LARGE \Examplefontiii A B C E F }
\vspace{2mm}

\scalebox{1.5}{\LARGE \Examplefontiii G H I J K}
\vspace{2mm}

\scalebox{1.5}{\LARGE \Examplefontiii L M N O P}

} Under Linux you may add fonts system-wide by copying the corresponding \verb|.ttf|-file to the 

{\centering \verb|/usr/share/fonts/truetype/|\par}

or via font-manager.

Here the following fonts were used:
\begin{itemize}
	\item[\Large \Examplefontiii W] \href{https://www.fontspace.com/tosca-font-f10033}{Tosca-8Rln}
	\item[\Large \Examplefontiii X] \href{https://www.fontspace.com/medievalalphabet-font-f13634}{Medievalalphabet-4EY6}
	\item[\Large \Examplefontiii Y] \href{https://www.fontspace.com/spanish-army-shields-two-font-f16698}{SpanishArmyShieldsTwo-g9aE}
	\item[\Large \Examplefontiii Z] \href{https://www.fontspace.com/hansschoenspergerrandomish-font-f8903}{Hansschoenspergerrandomish-VW6B}.
\end{itemize}

{\reasonablefont The font \say{Garamond}, however, is a very down-to-earth alternative. Creative, but professional. For document-wide use simply go for 
	
	{\centering \verb|\usepackage{ebgaramond}|.\par}

}
Math-fonts are a topic for another time. Obvious choices are 

{\centering \verb|eulervm|, \verb|mathpazo|, and \verb|fourier|.\par} 

\lettrine[lines=3, loversize=0.1]{\textcolor{black}{\Examplefontiv W}}{e} conclude this digression with another pearl of \href{https://www.ziereis-faksimiles.de/}{Initialkunst} and the last of the above four example fonts.
